% This is samplepaper.tex, a sample chapter demonstrating the
% LLNCS macro package for Springer Computer Science proceedings;
% Version 2.20 of 2017/10/04
%
\documentclass[runningheads]{llncs}
%
\usepackage{graphicx}
% Used for displaying a sample figure. If possible, figure files should
% be included in EPS format.
%
% If you use the hyperref package, please uncomment the following line
% to display URLs in blue roman font according to Springer's eBook style:
% \renewcommand\UrlFont{\color{blue}\rmfamily}

\begin{document}
%
\title{Economía Circular para combatir el cambio climático}
%
%\titlerunning{Abbreviated paper title}
% If the paper title is too long for the running head, you can set
% an abbreviated paper title here
%
\author{García Mariano\inst{1}\orcidID{0009-0002-5359-3519} \and
Jara Agustín\inst{1}\orcidID{0009-0005-9350-7955} \and
Ortíz Candelaria\inst{1}\orcidID{0009-0002-5359-3519} \and
Scacciante Genaro\inst{1}\orcidID{0009-0005-9350-7955}}
%
\authorrunning{Facultad de Ingeniería UNCuyo.}
% First names are abbreviated in the running head.
% If there are more than two authors, 'et al.' is used.
%
\institute{Universidad Nacional de Cuyo, ARG \
\url{https://www.uncuyo.edu.ar/}}
%
\maketitle{Técnicas y Herramientas Modernas I} % typeset the header of the contribution
%
\begin{abstract}
En el presente trabajo se desarrollará la implementación de la Economía Circular con el fin de combatir el cambio climático. Para ello se definirá y comparará esta reciente ideología económica con la tradicional y vigente  Economía Lineal. Luego se enunciarán los desafíos que presenta Argentina con el cambio climático y como se conectan con la economía Nacional.
También se mencionarán algunas medidas para promover la economía circular  en Argentina y algunos casos de éxito en la implementación de esta economía sostenible.
Finalmente luego de mencionar los desafios  que presenta Argentina en la ejecución de la Economía Circular, se presentaran medidas para vencer estas barreras. 


\keywords{Cambio Climático  \and Residuos \and Economía Circular \and Argentina.}
\end{abstract}
%
%
%
\section{Introducción}

\hspace{0.5cm} Para comenzar se definirá a la economía circular como un modelo de producción y consumo que implica compartir, alquilar, reutilizar, reparar, renovar y reciclar materiales y productos existentes todas las veces que sea posible para crear un valor añadido. De esta forma, el ciclo de vida de los productos se extiende.

En la práctica, implica reducir los residuos al mínimo. Cuando un producto llega al final de su vida, sus materiales se mantienen dentro de la economía siempre que sea posible gracias al reciclaje. Estos pueden ser productivamente utilizados una y otra vez, creando así un valor adicional.
Contrasta con el modelo económico lineal tradicional, basado principalmente en el concepto “usar y tirar” (obsolescencia programada) que requiere de grandes cantidades de materiales y energía baratos y de fácil acceso.

El calentamiento global debido al uso abusivo de los combustibles fósiles y a un modelo económico de consumismo de usar y tirar, además del crecimiento del número de individuos de la especie –cerca de 8000 millones-, nos ha llevado al actual cambio climático, que se caracteriza especialmente por la rapidez a la que se está produciendo. El clima ya está siendo más cálido, más imprevisible; con olas de calor intensas, inundaciones, reducción de caudales de los ríos, etc. 

Los combustibles fósiles nos han permitido disponer de energía accesible, intensa y barata, permitiendo un crecimiento exponencial. Hemos gastado en 270 años la que el planeta había empleado más de 30 millones en acumular. El modelo económico basado en los combustibles fósiles ha llegado a sus límites. En términos generales, una parte importante de la humanidad está viviendo mucho mejor que antes del año 1750. Pero todo tiene un coste, y este es el calentamiento global y el uso intensivo de los recursos del planeta. Un calentamiento relativamente pequeño en términos de temperatura, pero enorme en términos de la energía añadida a la atmósfera, a los océanos, y a los hielos. Los océanos, que cubren el 70 \% de la superficie del planeta, ya acumulan el 90 \% del calor retenido por los gases de efecto invernadero en ese corto período de años.
\section{Consecuencias del uso de la Economía Lineal Tradicional}

\hspace{0.5cm} Con respecto a los efectos físicos y sociales del cambio climático, las ciudades sufrirán los efectos de forma particularmente extrema, y especialmente las situadas en las costas que concentran cerca del 40 \% de la población mundial. Además, los efectos sobre los diferentes ecosistemas naturales y artificiales, también alteraran las producciones agrícolas y ganaderas de las que depende nuestra alimentación. Generando migraciones, cambios sociopolíticos y económicos.

Ante este escenario, el actual modelo socio-económico-energético dominante, genera la emisión de gases de efecto invernadero que provocan el actual cambio climático y que son producto de nuestra economía extractiva “extraer-producir-tirar”. Se debe cambiar radicalmente la forma en la que generamos energía, nos movemos, producimos y consumimos bienes y servicios  antes de 2050, y mucho mejor acelerar los cambios antes del 2030. Debemos movernos a energías renovables, electrificar el transporte, replantear los usos del agua, reducir al mínimo los residuos generados, apostar por la agricultura ecológica e impulsar la fiscalidad verde. Es decir, ir a un modelo de economía circular. Acelerar el cambio hacia una economía circular es esencial para alcanzar los objetivos climáticos. Es clave para aumentar la resiliencia ante la emergencia climática.

La economía circular juega un papel determinante. Y debe reducirse cuanto antes las emisiones de gases de efecto invernadero en todos los sectores, ya que es un componente esencial para lograr la neutralidad climática. La extracción y el procesamiento de los recursos naturales causan la mitad de las emisiones mundiales y más del 90 \% de la pérdida de biodiversidad. La economía circular implica evitar el consumo innecesario, los residuos y el uso de combustibles fósiles mediante la reutilización, la reparación y el reciclaje de los materiales y productos existentes en toda la cadena de valores: extracción y uso de recursos, producción, distribución, uso y materiales residuales. 

La transición hacia una economía circular nos debería permitir alcanzar las necesidades de una población en crecimiento, mientras avanzamos hacia una economía prospera y resilente, que puede funcionar a largo plazo, de una forma sostenible.

Es un proceso inevitable de transformación, con sus necesidades, objetivos y sus costes, y con una elevada probabilidad de que provoque protestas, tanto por la defensa de intereses creados como por la inercia a los cambios. Estamos ante una transformación del modelo con un gran impacto social, por agotamiento del modelo basado en el uso intensivo de los recursos, será necesario un nuevo contrato social, para que la justicia social esté también en el centro del proceso de cambio. Sin reducir la competitividad social, sin dejar a nadie atrás.

\section{Desafíos del cambio climático en Argentina}


\hspace{0.5cm} 1. Variabilidad climática y eventos extremos: Argentina ha experimentado un aumento en la frecuencia e intensidad de eventos climáticos extremos, como sequías, inundaciones y tormentas.


2. Disminución de los recursos hídricos: El cambio climático afecta el ciclo hidrológico, lo que se traduce en una disminución de los recursos hídricos disponibles. Esto tiene un impacto directo en la disponibilidad de agua para consumo humano, agricultura, generación de energía hidroeléctrica y mantenimiento de los ecosistemas acuáticos.


3. Pérdida de biodiversidad: La biodiversidad en Argentina se ve amenazada por el cambio climático, ya que muchas especies enfrentan desafíos para adaptarse a las nuevas condiciones. La pérdida de hábitats, la modificación de los patrones de migración y la extinción de especies son consecuencias preocupantes. La biodiversidad es esencial para la salud de los ecosistemas y para el suministro de servicios ecosistémicos vitales.

\section{Conexión entre el cambio climático y la economía nacional de Argentina}

\hspace{0.5cm} El cambio climático tiene un impacto directo en el sector agrícola de Argentina, que es uno de los pilares de su economía. Los cambios en los patrones de lluvia, las temperaturas extremas y la mayor frecuencia de eventos climáticos extremos pueden afectar la productividad y la calidad de los cultivos, lo que a su vez reduce la producción agrícola y, potencialmente, los ingresos de los agricultores, también las exportaciones se verían notablemente reducidas. Además, la disminución de los recursos hídricos debido al cambio climático puede limitar la disponibilidad de agua para riego, lo que agrava aún más los desafíos en el sector agrícola.
En relación a los recursos naturales, Argentina es rico en varios de ellos, como el agua, los bosques y los minerales. Los efectos negativos pueden ser por ejemplo, la disminución de las reservas de agua debido a la reducción de las precipitaciones y el derretimiento de los glaciares puede afectar la disponibilidad de agua dulce para consumo humano, riego agrícola y generación de energía hidroeléctrica. Además, los cambios en los patrones climáticos pueden tener un impacto en los ecosistemas forestales, aumentando el riesgo de incendios forestales y la pérdida de biodiversidad.

Cabe destacar la presencia del turismo, debido a que Argentina es un destino turístico importante. El cambio climático puede afectar la atracción turística y los ingresos asociados debido a la degradación de biodiversidad marítima lo que puede afectar el turismo de buceo y snorkel en la costa argentina; además se puede producir la erosión costera y pérdida de playas.

El cambio climático también puede tener impactos significativos en la infraestructura ya que los eventos climáticos extremos, como inundaciones y tormentas intensas, pueden causar daños a carreteras, puentes y edificios. La reconstrucción y reparación de esta infraestructura pueden representar costos significativos para el gobierno y la economía nacional. Además, las inversiones necesarias para adaptar la infraestructura a los efectos del cambio climático, como la construcción de sistemas de drenaje más eficientes o la protección costera, también pueden implicar costos considerables.
Por último puede impactar en el sector energético de Argentina. A medida que aumenta la conciencia sobre la necesidad de reducir las emisiones de gases de efecto invernadero, existe una mayor presión para la transición hacia fuentes de energía más limpias y renovables. Es complicado a causa de que la matriz energética de Argentina, que ha dependido tradicionalmente de los combustibles fósiles, como el petróleo y el gas. 


\section{Medidas para promover la economía circular en Argentina}

\hspace{0.5cm} Argentina posee una ley de  desarrollo y promoción de la economía circular. En el capítulo II de la misma se especifican acciones, sujetos alcanzados y responsabilidades. Dentro de él se mencionan una serie de acciones a realizar:

\textbf{Segregación binaria de residuos domiciliarios}. Las provincias, sus municipios y la Ciudad Autónoma de Buenos Aires, deberán adoptar progresivamente un sistema de gestión que asegure, al menos, una segregación binaria de los residuos domiciliarios generados en sus territorios, promoviendo una disposición inicial selectiva y posterior recolección diferenciada que contemple, por un lado, los residuos reciclables secos, y, por otro, los residuos considerados basura, procediéndose a su distinción mediante el uso de distintos colores.
 
\textbf{Código armonizado de colores para residuos}. Las provincias, sus municipios y la Ciudad Autónoma de Buenos Aires, deberán implementar progresivamente el código armonizado de colores para la identificación, clasificación y segregación de residuos domiciliarios, de acuerdo con lo que establezca la reglamentación de la presente ley.

\textbf{Recuperación y valorización de residuos}. Las autoridades competentes de cada jurisdicción deben adoptar las medidas necesarias para promover las actividades de recuperación y valorización de residuos, destinadas a empresas, cooperativas y organizaciones sociales que directa o indirectamente intervienen en las distintas etapas de producción, en cada una de las ramas de actividad o sectores de la economía.

\textbf{Incorporación de material reciclado a la producción}. Los procesos de producción de bienes en la República Argentina deberán incorporar, siempre que estén disponibles en cantidades suficientes en el mercado y exista tecnología adecuada, el mayor componente de materiales reciclados posibles. La autoridad de aplicación deberá establecer gradualmente cada cuatro (4) años los porcentajes de materiales reciclados a utilizar en cada proceso industrial, para lo que deberá previamente realizar un estudio de las previsiones de materiales reciclables disponibles en el mercado, la disponibilidad de la tecnología necesaria y la viabilidad económica financiera.

En el capítulo IV Educación para la economía circular se establece que Las autoridades competentes deberán coordinar con los consejos federales de Medio Ambiente (COFEMA) y de Cultura y Educación la implementación de planes y programas en los sistemas de educación formal y no formal tendientes a la difusión de la Economía Circular y sus implicancias positivas para el ambiente, la economía y la sociedad. Los contenidos básicos a impartir deberán incluir como mínimo: los principios de la Economía Circular y su aplicación, ecodiseño, ciclo de vida del producto, producción y consumo sustentable y la educación sobre las 7R (rediseñar, reducir, reutilizar, reparar, renovar, recuperar y reciclar).

En el capítulo V Promoción de la innovación industrial se establece que se les otorgará créditos para estimular e incentivar proyectos de inversión de adaptación a la Economía Circular así como el desarrollo de programas de innovación.

Finalmente el capítulo VI Incentivos fiscales menciona las siguientes medidas:

- Reducción de contribuciones patronales

- Exención frente al IVA

- Amortización en el Impuesto a las Ganancias

\section{Casos exitosos de economía circular}

\hspace{0.5cm} \textbf{1. Programa "Basura Cero" en la Ciudad de Buenos Aires:}

En la Ciudad de Buenos Aires la basura es uno de los problemas estructurales más importantes. La ley 1854 de Gestión de los Residuos Urbanos (Basura Cero), sancionada en 2005, propone la progresiva reducción de la cantidad de basura que se entierra mediante el crecimiento de las industrias asociadas al reciclado y la reducción en la generación de residuos.

Los objetivos principales que se llevaron a cabo son: 

- Concientizar a los vecinos y a los grandes generadores acerca de la necesidad de la separación en origen de residuos, diferenciando entre reciclables y basura.

- Minimización de los residuos a enterrar mediante la consolidación de práctica de separación de materiales reciclables en origen.

- Formalización e integración de los Recuperadores Urbanos en el circuito del servicio público de recolección diferenciada.

- Proyectos ambientales que contemplan la puesta en marcha de sistemas de recuperación y reciclado de residuos sólidos urbanos.

- Aumento de los materiales que regresan como materia prima post consumo a la industria.

- Contribuir al ordenamiento de la cadena de valor del reciclado.

A más de catorce años desde su sanción, la ley sigue sin respetarse.

Los residuos sólidos urbanos aún constituyen una de las problemáticas más graves de la Ciudad. El gobierno sigue enviando la mayor parte de la basura a rellenos sanitarios, cuando la meta fijada por la ley era una reducción escalonada del 30 \% para 2010, del 50 \% para 2012, y un 75 \% para 2017. No obstante, en 2017 se llegó a un 26 \% de reducción de residuos, es decir, no se cumplió ni siquiera el primer objetivo del plan.

Sumado a esto, la Legislatura de la Ciudad aprobó el 3 de mayo de 2018 la modificación de la Ley de Basura Cero para permitir la incineración de residuos, una tecnología obsoleta y contaminante, explícitamente prohibida en el texto original.

A través de la Ley N° 5.966 también se modificaron los objetivos y el año de base para reducir la basura que se envía a rellenos sanitarios: 50 \% para 2021, 65 \% para 2025 y 80 \% para 2030. Los legisladores tomaron
como base a 2012, año en el que se enterraron 2.131.072  de toneladas
de residuos, en vez de 2017, cuando la cantidad fue aproximadamente la mitad.

\textbf{2. Programa "Argentina Recicla":}

Este programa, impulsado por el Ministerio de Ambiente y Desarrollo Sostenible, busca promover la economía circular a través del fomento del reciclaje. En Argentina son miles los cartoneros, carreros y recicladores que recolectan y tratan grandes toneladas de materiales reciclables al día evitando su disposición final en los más de 5 mil basurales a cielo abierto existentes. Se establecen convenios con cooperativas de recuperadores urbanos para la recolección selectiva de residuos reciclables, brindándoles apoyo técnico y capacitación. A su vez, se promueve la compra de productos reciclados por parte del gobierno y se implementan políticas de responsabilidad extendida del productor.

Se enfoca en cuatro líneas de intervención:

Recicladores de Base:promoción de derechos garantizando condiciones mínimas de seguridad en el trabajo, infraestructura y equipamiento. Se facilita ropa de trabajo, bolsones, carros, elementos de protección personal y herramientas.

Sistemas Locales de Reciclado: infraestructura, equipamiento, articulación y asistencia técnica a todos los actores de un sistema local de reciclado en todas las etapas.

Redes de Comercialización: se brinda asistencia técnica e infraestructura para mejorar la logística y comercialización colectiva de los materiales reciclables.

Valorización de otras corrientes de residuos: Asistencia técnica y equipamiento para desarrollo del valor agregado de los reciclables y el tratamiento de nuevas corrientes de residuos como Orgánicos, Residuos de aparatos eléctricos y electrónicos (RAEE), y Neumáticos fuera de uso (NFU), entre otros.

\textbf{3. Iniciativas de economía circular en el sector textil:}

En Argentina, han surgido diversas iniciativas para fomentar la economía circular en el sector textil. Por ejemplo, empresas como "La Laucha Ecoalfombra" fabrican alfombras utilizando materiales reciclados, como redes de pesca y neumáticos desechados. También existen empresas textiles que utilizan tejidos reciclados o promueven la reparación y el intercambio de prendas para alargar su vida útil.

\textbf{4. Programa "Banco de Alimentos":}

Esta organización sin fines de lucro se dedica a recuperar alimentos que están en buen estado pero que de otra manera serían desperdiciados. Estos alimentos son distribuidos a comedores comunitarios y organizaciones benéficas, evitando su desperdicio y brindando apoyo a personas en situación de vulnerabilidad.

Aproximadamente un tercio de las emisiones globales de gases de efecto invernadero se atribuyen a los sistemas alimentarios, lo que incluye las emisiones generadas por el uso de la tierra, la producción agrícola, la cadena de suministro de alimentos, el desperdicio de alimentos y más. Esto, a su vez, empeora el cambio climático y dificulta el cultivo y el acceso a los alimentos.

Dado que aproximadamente 828 millones de personas se enfrentan al hambre en la actualidad, el impacto del cambio climático en nuestros sistemas alimentarios es una amenaza grave y catastrófica. Sin embargo, los bancos de alimentos dirigidos por la comunidad en todo el mundo están abordando estos dos problemas y son una solución ecológica para mejorar el acceso a los alimentos y mitigar el cambio climático.

Al recuperar alimentos que de otro modo podrían haberse perdido o desperdiciado, los bancos de alimentos están evitando miles de millones de kilogramos de gases de efecto invernadero (GEI) entren en la atmósfera.
Los miembros del Banco de Alimentos, en casi 50 países, recuperaron 514.537 toneladas de alimentos excedentes y los distribuyeron a 39 millones de personas. Si el mismo volumen de alimentos hubiera ido al vertedero, su descomposición habría emitido aproximadamente 1,7 millones de toneladas de CO2 equivalente, una proporción significativa del cual habría sido metano, un potente gas de efecto invernadero.

Las emisiones ahorradas suponen un incremento respecto a 2019, cuando los miembros de la Red evitaron 1.487 millones de kilogramos de CO2 a través de la redirección de alimentos.

\textbf{5. La economía circular en la industria del papel: el caso de SAICA}

En este apartado, se va a exponer el caso de una de las empresas aragonesas más conocidas, como es la multinacional zaragozana S.A Industrias Celulosa Aragonesa (SAICA), que lidera el sector en España y es el tercer actor más destacado de este mercado a nivel europeo, generando 9.000 puestos de trabajo y fabricando 2,5 millones de toneladas de papel anuales, lo cual significó una ganancia de más de 2.500 millones de euros. La compañía tiene como objeto principal desarrollar y producir “soluciones sostenibles para el embalaje de papel y cartón ondulado y su posterior recuperación”.

Según Greenpeace, “la industria del papel se ubica al tope del ranking en materia de uso de recursos naturales y generación de contaminantes, todo para fabricar un producto que es usualmente descartado inmediatamente. 

De esta manera, SAICA, ha revolucionado la industria del papel y ha conseguido, a través de la innovación tecnológica, optimizar los recursos, reducir el uso de agua en la fabricación de papel y valorizar los recursos. SAICA sigue un proceso para desarrollar su actividad por el cual, en primer lugar, recupera a través de su sección de medio ambiente NATUR, papel, plástico y cartón en los diferentes donde está presente, posteriormente recicla el material recuperado y, así, finalmente, reduce la cantidad de materia prima necesitada. Es un proceso circular que está en constante funcionamiento.

SAICA ha ido evolucionando a lo largo del tiempo sus sistemas de fabricación, pasando de la celulosa al papel reciclado, incorporando plantas de cogeneración en sus instalaciones e incorporando plantas de tratamiento de aguas de proceso en las que se produce biogás, aprovechandolo para producir electricidad y vapor y utilizando los residuos de las fábricas para producción eléctrica en la Planta de Valorización Energética.

Estos ejemplos destacan el compromiso y los esfuerzos de diversos actores tanto en Argentina como en otros países para implementar prácticas de economía circular. 

\section{Desafíos y barreras para la implementación de la Economía Circular en Argentina}

\hspace{0.5cm} \textbf{1. Falta de conciencia y comprensión:} Uno de los principales desafíos es la falta de conciencia y comprensión generalizada sobre la importancia y los beneficios de la economía circular. Muchas empresas y consumidores aún no comprenden completamente los conceptos y las implicaciones de la economía circular, lo que dificulta su adopción y aplicación en la práctica. Es necesario aumentar la conciencia y la educación sobre la economía circular a través de campañas de sensibilización, programas de capacitación y difusión de información.


\textbf{2. Falta de políticas y regulaciones adecuadas:} La falta de políticas y regulaciones claras y específicas que fomenten la economía circular es otro desafío importante. La ausencia de un marco legal sólido puede dificultar la implementación de prácticas circulares y la inversión en infraestructura necesaria. Es necesario que los gobiernos y las instituciones desarrollen marcos normativos que promuevan la economía circular, estableciendo metas y objetivos claros, proporcionando incentivos y apoyo financiero, y estableciendo estándares de calidad y responsabilidad.


\textbf{3. Limitaciones tecnológicas y de infraestructura:} La implementación de la economía circular a menudo requiere tecnologías y procesos innovadores, así como una infraestructura adecuada para el tratamiento y la valorización de los residuos y subproductos. En Argentina, puede existir una falta de acceso a tecnologías avanzadas y una infraestructura insuficiente para la gestión adecuada de los recursos y residuos. Es necesario fomentar la investigación y el desarrollo de tecnologías circulares, así como invertir en infraestructuras de reciclaje, tratamiento de residuos y energías renovables.


\textbf{4. Barreras económicas y financieras:} La adopción de prácticas de economía circular puede requerir inversiones significativas y a largo plazo, lo que puede representar una barrera económica para las empresas, especialmente las pequeñas y medianas. El acceso a financiamiento y a programas de apoyo específicos para la transición a la economía circular es fundamental para superar esta barrera.

\section{Medidas para implementar la economía circular en Argentina}

\hspace{0.5cm} \textbf{Educación y sensibilización:} Implementar programas educativos y campañas de sensibilización a nivel nacional para promover la comprensión y conciencia sobre la economía circular. Esto puede incluir la integración de la educación sobre economía circular en los planes de estudio escolares, la organización de talleres y conferencias, y la difusión de información a través de medios de comunicación y plataformas digitales.

\textbf{Marco regulatorio y político favorable:} Establecer políticas y regulaciones claras y específicas que promuevan la economía circular. Esto puede incluir la implementación de metas y objetivos de economía circular, la creación de incentivos fiscales y financieros para las empresas que adopten prácticas circulares, y el establecimiento de estándares de calidad y responsabilidad extendida del productor. 

\textbf{Apoyo financiero y acceso a la inversión:} Establecer programas de financiamiento y acceso a la inversión específicos para proyectos y empresas relacionados con la economía circular. Esto puede incluir la creación de fondos de inversión sostenible, la provisión de préstamos a tasas preferenciales para proyectos circulares y la facilitación del acceso a subvenciones y subsidios para empresas que implementen prácticas circulares.

\textbf{Colaboración público-privada:} Fomentar la colaboración entre el sector público y privado para impulsar la economía circular. Esto puede incluir la creación de alianzas estratégicas entre el gobierno y las empresas para el desarrollo de proyectos y programas conjuntos, la promoción de la simbiosis industrial y la colaboración en la investigación y desarrollo de tecnologías circulares.

\textbf{Innovación y desarrollo tecnológico:} Fomentar la investigación y el desarrollo de tecnologías y soluciones innovadoras que impulsen la economía circular. Esto puede incluir la creación de programas de financiamiento para la investigación científica y el desarrollo tecnológico en áreas clave, como el reciclaje avanzado, la valorización de residuos y la producción de materiales sostenibles.

\textbf{Alianzas internacionales y cooperación:} Promover la cooperación internacional y el intercambio de buenas prácticas en economía circular. Argentina puede aprender de los casos exitosos y las experiencias de otros países que han avanzado en la implementación de la economía circular. Esto puede incluir la participación en redes internacionales, la colaboración en proyectos de investigación conjuntos y el intercambio de conocimientos a través de programas de capacitación y visitas técnicas.

\section{Conclusión}

\hspace{0.5cm} A lo largo de este trabajo se han definido los conceptos más importantes sobre economía circular, como se diferencia de la economía lineal tradicional, como así también, el impacto de la misma en distintos sectores como el industrial, económico y social.

En base a estos conceptos se analizó la economía circular como una herramienta para hacer frente a los problemas del cambio climático, se estudió un proyecto de ley de  desarrollo y promoción de la economía circular donde se enunciaron algunas medidas para fomentar la implementación de este nuevo modelo económico. 

Luego se presentaron distintos casos exitosos de economía circular en Argentina y el mundo, lo cual permitió visualizar los avances del país respecto a la economía sostenible y la lucha contra el cambio climático.

Por último, identificadas las barreras a las cuales se enfrenta día a día la Argentina en la implementación de la economía circular, se presentaron una serie de soluciones para superar dichos obstáculos y así finalmente reducir los efectos adversos del cambio climático.

Si bien en Argentina se han desarrollado numerosas medidas para fomentar la economía circular, queda por realizar un gran avance que permita contrarrestar los efectos del cambio climático. Existen varias barreras que dificultan la implementación en Argentina, principalmente económicas y culturales.

Finalmente, superar estos desafíos y barreras requerirá una colaboración entre el gobierno, las empresas y la sociedad civil. Es fundamental establecer una visión clara y comprometida, así como una estrategia integral que aborde estos desafíos y promueva la transición hacia una economía circular en Argentina.

\begin{thebibliography}{8}
\bibitem{ref_article1}
https://www.europarl.europa.eu/news/es/headlines/economy/20151201STO05603/economia-circular-definicion-importancia-y-beneficios

\bibitem{ref_lncs1}
https://www.unido.org/our-priorities/climate-action

\bibitem{ref_book1}
https://economiacircular.org/la-economia-circular-es-clave-para-combatir-el-actual-cambio-climatico/

\bibitem{ref_proc1}
Juan Mario Pais, Senador de La Nacion Argentina, Ley de Desarrollo y Promoción de la Economía Circular en Argentina, Buenos Aires, Argentina (2021)

\bibitem{ref_url1}
https://www.infobae.com/america/ciencia-america/2019/08/10/los-desafios-de-la-argentina-frente-al-calentamiento-global-entre-la-inaccion-y-la-necesidad-de-cambios-sin-precedentes/

\bibitem{ref_url1}
https://www.lanacion.com.ar/economia/industria-moda-como-construir-economia-circular-nid2406506/

\bibitem{ref_url1}
https://www.greenpeace.org/argentina/involucrate/con-incineracion-no-hay-basura-cero/ley-de-basura-cero-mas-de-14-anos-de-metas-incumplidas/

\bibitem{ref_url1}
https://www.argentina.gob.ar/desarrollosocial/argentinarecicla

\bibitem{ref_url1}
https://www.foodbanking.org/es/resources/food-banks-for-people-and-the-planet/

\bibitem{ref_url1}
https://eco-circular.com/2019/11/26/siete-casos-de-exito-de-la-economia-circular/

\end{thebibliography}
\end{document}
